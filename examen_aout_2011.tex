\documentclass[a4paper, 12pt]{article}

\usepackage[francais]{babel} % Document en français
\usepackage[utf8]{inputenc} % Document au format UTF8
\usepackage[T1]{fontenc} % Suppression d'un warning pour la langue francaise
\usepackage{amsmath} % Pour les équations
\usepackage{textcomp} % Utilisation du symbole degré
\usepackage[left=2.2cm,right=2.2cm,top=2.5cm,bottom=2.5cm]{geometry} % Mise en page

\author{durée : 3 heures}
\title{Examen BIR1304 \\ Probabilités et statistique (II)}
\date{le 24 août 2011}

\begin{document}
\maketitle

  \section*{Question 1}
  
  Un agriculteur possède un champ carré de longueur $\mu$, dont il désire estimer la surface. Pour cela, il effectue 2 mesures indépendantes $X_{1}$ et $X_{2}$ de la longueur de son chmpa telles que ces mesures suivent une loi normale d'espérance $\mu$ et de variance $\sigma^{2}$. Pour estimer la surface, plusieurs possibilités s'offrent à lui. Il peut faire la moyenne des mesures et l'élever au carré, ou bien il peut élever les mesures au carré et prendre leur moyenne. Une troisième solution revient à multiplier les 2 mesures. Les trois estimateurs correspondants sont les suivants :
  
  \[
   \hat{\theta_{1}} = \frac{(X_{1} + X_{2})^{2}}{4}
  \]

  \[
   \hat{\theta_{2}} = \frac{(X_{1}^{2} + X_{2}^{2})}{2}
  \]
  
  \[
   \hat{\theta_{3}} = X_{1} * X_{2}
  \] \bigskip
  
  Calculez le biais de ces 3 estimateurs.
  
  \section*{Question 2}
  
  Un chercheur étudie l'effet d'un raccourcisseur de paille sur les céréales. Un raccourcisseur de paille est une molécule, fort utilisée en agriculture, qui agit sur la croissance de la plante afin de limiter son développement en hauteur. Dans une expérience, il étudie la différence entre la hauteur des plantes réparties entre 2 groupes. Le premier groupe n'est pas traité et sert de témoin. Le second groupe est traité à l'aide du produit. A la fin de la période de croissance, le chercheur a mesuré la hauteur (en mètres) de quelques plantes choisies au hasard dans chaque groupe. Voici ses résultats : \bigskip
  
  \begin{tabular}{l|cccccccc}
      Plante n\textdegree & 1 & 2 & 3 & 4 & 5 & 6 & 7 & 8 \\
      \hline
      Témoin & 1.37 & 1.01 & 1.36 & 1.48 & 1.34 & 1.46 & 1.21 & 1.43 \\
      Traités & 1.21 & 1.00 & 1.14 & 1.33 & 1.07 & 1.10 & 1.04 & 1.28 \\
  \end{tabular} \bigskip
  
  \begin{enumerate}
   \item Donnez un intervalle de confiance à la hauteur moyenne d'une plante non traitée.
   \item La variance de la hauteur peut-elle être considérée comme identique entre les plantes témoins et les plantes traitées ?
   \item Le traitement induit-il une diminution de la taille des plantes ? Justifiez.
  \end{enumerate}
  
  Pour chacun des tests et intervalles de confiance, considérez le niveau de confiance 1 - $\alpha$ = 0.95.
  
  \section*{Question 3}
  Un scientifique fait des recherches sur l'échauffement des cellules lorsqu'elles sont soumises à des ondes électromagnétiques. Il a soumis des cellules à des ondes de fréquences croissantes et a obtenu les résultats suivants : \bigskip
  
  \begin{tabular}{lcccccccc}
   Echauffement (mK) & 0.06 & 0.05 & 0.06 & 0.36 & 0.36 & 0.54 & 0.37 & 0.61 \\
   \hline
   Fréquence (GHz) & 0.3 & 0.5 & 0.7 & 0.9 & 1.1 & 1.3 & 1.5 & 2 \\
  \end{tabular} \bigskip
  
  Un modèle linéaire sous forme de droite (pente + intercept) est envisagé pour décrire l'effet de la fréquence sur l'échauffement. \bigskip
  
  \begin{enumerate}
   \item Ecrivez ce modèle sous forme matricielle et estimez-en les paramètres.
   \item Calculer un intervalle de confiance pour la pente.
   \item D'après le modèle utilisé, la fréquence a-t-elle un impact sur l'échauffement des cellules ?
   \item Durant l'expérience, la puissance de l'onde a également été mesurée. En considérant une extension du modèle précédent sous forme d'un plan, le paramètre supplémentaire étant associé à la puissance, la SCR de ce nouveau modèle est de 0.05. Ce modèle est-il meilleur que le précedent ?
  \end{enumerate}


\end{document}
